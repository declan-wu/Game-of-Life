\documentclass[12pt]{article}

\usepackage{graphicx}
\usepackage{paralist}
\usepackage{amsfonts}
\usepackage{amsmath}
\usepackage{hhline}
\usepackage{booktabs}
\usepackage{multirow}
\usepackage{multicol}
\usepackage{url}

\oddsidemargin -10mm
\evensidemargin -10mm
\textwidth 160mm
\textheight 200mm
\renewcommand\baselinestretch{1.0}

\pagestyle {plain}
\pagenumbering{arabic}

\newcounter{stepnum}

%% Comments

\usepackage{color}

\newif\ifcomments\commentstrue 

\ifcomments
\newcommand{\authornote}[3]{\textcolor{#1}{[#3 ---#2]}}
\newcommand{\todo}[1]{\textcolor{red}{[TODO: #1]}}
\else
\newcommand{\authornote}[3]{}
\newcommand{\todo}[1]{}
\fi

\newcommand{\wss}[1]{\authornote{blue}{SS}{#1}}

\title{COMPSCI 2ME4 A4 Specification}
\author{Song Tao Wu, wus92}

\begin {document}

\maketitle
This Module Interface Specification (MIS) document contains modules, types and
methods for implementing the state of a game called Game of Life. \\ \\
\begin{Large}
The Rules for Game of Life: \\
\end{Large}
For a space that is 'populated':

Each cell with one or no neighbors dies, as if by solitude.

Each cell with four or more neighbors dies, as if by overpopulation.

Each cell with two or three neighbors survives. \\
For a space that is 'empty' or 'unpopulated': 

Each cell with three neighbors becomes populated.


\bibliographystyle{plain}
\bibliography{SmithCollectedRefs}

\newpage

\section* {Grid Types Module}

\subsection*{Module}

GridTypes

\subsection* {Uses}

N/A

\subsection* {Syntax}

\subsubsection* {Exported Constants}

row\_limit = 15\\
column\_limit = 15
 
\subsubsection* {Exported Types}

GridT = sequence of [row\_limit] [column\_limit] of $\mathbb{B}$.

\subsubsection* {Exported Access Programs}

None

\subsection* {Semantics}

\subsubsection* {State Variables}

None

\subsubsection* {State Invariant}

None

\newpage

\section* {File Utility Module}

\subsection*{Library}

FileUtil

\subsection* {Uses}

Grid Types \\
Game Board ADT

\subsection* {Syntax}

\subsubsection* {Exported Constants}

N/A
 
\subsubsection* {Exported Types}

N/A

\subsubsection* {Exported Access Programs}

\begin{tabular}{| l | l | l | l |}
\hline
\textbf{Routine name} & \textbf{In} & \textbf{Out} & \textbf{Exceptions}\\
\hline
readFile  & String & GridT & file\_not\_find \\
\hline
writeToFile & GameBoard, String  &  & \\
\hline
\end{tabular}


\subsection* {Semantics}

\subsubsection* {State Variables}

None

\subsubsection* {State Invariant}

None

\subsubsection* {Access Routine Semantics}

\noindent readFile(fileName):
\begin{itemize}
\item transition: read data from the file associated with the string fileName.
  Use this data to construct a new gridT, which will be used to construct the new gameboard.\\

  The text file has the following format, where $C_{xy}$ stand for strings that
  represent the state of each cell. The state is
  represented by either the string ``0'' or ``*'' which represents true(occupied) and false(unoccupied) respectively. All data values in a row are separated by spaces.  Rows are
  separated by a new line.  
  \begin{equation}
    \begin{array}{ccccccc}
      c_{00}, & c_{01}, & c_{02}, &..., & c_{0column\_limit} \\
      c_{10}, & c_{11}, & c_{12}, &..., & c_{1column\_limit} \\
      ..., & ..., & ..., & ..., & ... \\
      c_{row\_limit0}, & c_{row\_limit1}, & c_{row\_limit2}, &..., & c_{row\_limit\ column\_limit} \\
    \end{array}
  \end{equation}

\item out := G where $\forall\, x, y: \mathbb{N} | x \in [0..\text{row\_limit}] \wedge y \in [0..\text{column\_limit}]:  G[x][y] :=  c_{xy})$ \\

\item exception: exc := (fileName does not exist $\implies$ file\_not\_find)

\end{itemize}

\noindent writeToFile(board, fileName):
\begin{itemize}
\item transition: write the grid data from board to the file associated with the string fileName. If the file does not exist, create a new file.\\

  The text file has the following format, where $(\forall\, x, y: \mathbb{N} | x \in [0..\text{row\_limit}] \wedge y \in [0..\text{column\_limit}]: C_{xy} :=  \text{board.getGrid(x, y)})$ $C_{xy}$ stands for the character that represents the state of each cell. The state is
 represented by either the string ``0'' or ``*'' which represents true(occupied) and false(unoccupied) respectively. All data values in a row are separated by spaces.  Rows are
  separated by a new line.  

  \begin{equation}
    \begin{array}{ccccccc}
      c_{00}, & c_{01}, & c_{02}, &..., & c_{0column\_limit} \\
      c_{10}, & c_{11}, & c_{12}, &..., & c_{1column\_limit} \\
      ..., & ..., & ..., & ..., & ... \\
      c_{row\_limit0}, & c_{row\_limit1}, & c_{row\_limit2}, &..., & c_{row\_limit\ column\_limit} \\
    \end{array}
  \end{equation}


\item exception: none

\end{itemize}



\newpage

\section* {View Module}

\subsection*{Library}

View

\subsection* {Uses}

Grid Types \\
Game Board ADT

\subsection* {Syntax}

\subsubsection* {Exported Constants}

N/A
 
\subsubsection* {Exported Types}

N/A

\subsubsection* {Exported Access Programs}

\begin{tabular}{| l | l | l | l |}
\hline
\textbf{Routine name} & \textbf{In} & \textbf{Out} & \textbf{Exceptions}\\
\hline
display  & GameBoard & &  \\
\hline
\end{tabular}

\subsection* {Semantics}

\subsubsection* {State Variables}

None

\subsubsection* {State Invariant}

None

\subsubsection* {Access Routine Semantics}

\noindent display(board):
\begin{itemize}
\item transition: 
output the grid data from board to the standard output. \\
  The output has the following format, where $(\forall\, x, y: \mathbb{N} | x \in [0..\text{row\_limit}] \wedge y \in [0..\text{column\_limit}]: C_{xy} :=  \text{board.getGrid(x, y)})$ $C_{xy}$ stands for the character that represents the state of each cell. The state is
 represented by either the string ``0'' or ``*'' which represents true(occupied) and false(unoccupied) respectively. All data values in a row are separated by spaces.  Rows are
  separated by a new line.  

  \begin{equation}
    \begin{array}{ccccccc}
      c_{00}, & c_{01}, & c_{02}, &..., & c_{0column\_limit} \\
      c_{10}, & c_{11}, & c_{12}, &..., & c_{1column\_limit} \\
      ..., & ..., & ..., & ..., & ... \\
      c_{row\_limit0}, & c_{row\_limit1}, & c_{row\_limit2}, &..., & c_{row\_limit\ column\_limit} \\
    \end{array}
  \end{equation}

\item exception: none

\end{itemize}


\newpage


\section* {Game Board ADT Module}

\subsection*{Template Module}

GameBoard

\subsection* {Uses}

\noindent GridTypes

\subsection* {Syntax}

\subsubsection* {Exported Access Programs}

\begin{tabular}{| l | l | l | l |}
\hline
\textbf{Routine name} & \textbf{In} & \textbf{Out} & \textbf{Exceptions}\\
\hline
new GameBoard  & GridT & GameBoard & invalid\_argument\\
\hline
new GameBoard &  &  GameBoard & \\
\hline
next\_state & $\mathbb{N}$, $\mathbb{N}$  &  $\mathbb{B}$ & \\
\hline
num\_of\_alive\_neighbours & $\mathbb{N}$, $\mathbb{N}$  &  $\mathbb{B}$ & \\
\hline
getGrid &  & GridT & \\
\hline
update &  &  & \\
\hline
\end{tabular}

\subsection* {Semantics}

\subsubsection* {State Variables}

$grid$: GridT \textit{\# The game board grid}

\subsubsection* {State Invariant}

size(grid) = row$\_$limit $\times$ column$\_$limit


\subsubsection* {Assumptions \& Design Decisions}

\begin{itemize}

\item The BoardT constructor is called before any other access
  routine is called on that instance. Once a BoardT has been created, the
  constructor will not be called on it again.

\item For better scalability, this module is specified as an Abstract Data Type
  (ADT) instead of an Abstract Object. This would allow multiple games to be
  created and tracked at once by a client.

\end{itemize}


\subsubsection* {Access Routine Semantics}

\noindent GameBoard($\mathit{iniitial\_states}$):
\begin{itemize}
\item transition: 
$grid := \mathit{initial\_states} $
\item exception: $exc := ( \lnot \ (|$grid$| \  = \ $row$ \_$limit$ ) \implies $invalid$\_$argument$)$
\end{itemize}

\noindent GameBoard():
\begin{itemize}
\item transition: 
$grid$ := G where $(\forall\, x, y: \mathbb{N} | x \in [0..\text{row\_limit}] \wedge y \in [0..\text{column\_limit}]: G[x][y] = false)$
 
\item exception: none
\end{itemize}

\noindent next\_state($x, y$):
\begin{itemize}
\item output: 
$(\lnot grid[x][y] \implies \text{num\_of\_alive\_neighbours($x, y$)} = 3) \lor grid[x][y] \implies (\text{num\_of\_alive\_neighbours($x, y$)} = 3 \lor \text{num\_of\_alive\_neighbours($x, y$)} = 2) )$
\item exception: none

\end{itemize}

\noindent num\_of\_alive\_neighbours($x, y$):
\begin{itemize}
\item output : $ (+ \ a, b : \mathbb{N} \ | \ a \in [ max(x-1, 0)..min(x+1, \text{row\_limit})] \wedge b \in [ max(y-1, 0)..min(y+1, \text{column\_limit})] \wedge grid[a][b] : 1)$\\

\item exception: none

\end{itemize}

\noindent getGrid():
\begin{itemize}
\item output : grid

\item exception: none

\end{itemize}

\noindent update():
\begin{itemize}
\item transition: grid := G where 
$\forall\, x, y: \mathbb{N} | x \in [0..\text{row\_limit}] \wedge y \in [0..\text{column\_limit}]:  G[x][y] :=  next\_state($x, y$)$ \\

\item exception: none

\end{itemize}

\newpage

\section*{Critique of Design}

\subsection*{GridTypes Module}

\begin{itemize}
\item The module is just a library exporting constants like row size and column size and the whole gridT type - 2 dimensional vector of boolean value.

\item The constants and type will be used in all other files.  By defining those value and types here, I modularize the code to avoid the need to redefining those values in every other modules.

\end{itemize}

\subsection*{FileUtil Module}

\begin{itemize}
\item The module provides methods to write board to and read data from a file. There're two methods - readFile and writeToFile. Since both methods have to do with file utilities. This modularization shows the programming design principle - separation of concern. 

\item If we were to add more methods to interact with files, we could add the methods in this files. 

\end{itemize}

\subsection*{GameBoard Module}

\begin{itemize}
\item GameBoard provides methods to update the whole grid, and move on to the next state of the game.  

\item There are two constructors, one getter for state variable, and three methods to update the whole grid. the two methods - num\_of\_alive\_neighbours and next\_state are used to update the whole grid. The two methods could have been made private to enhance information hiding principle. But for unit testing purpose, I decided to make them public to give me more options and freedom to interact with the grid. 

\item The default constructor is used to set the whole grid to be unpopulated to avoid errors from not passing any parameters to the GameBoard. 

\end{itemize}

\subsection*{View Module}

\begin{itemize}
\item The module only provides one method to perform the basic view function.  It takes a GameBoard and output it to the standard output.  

\item the function could have been included in the GameBoard module. But for the sake of MVC - Module View Controller and principle of separation of concern, the display method is included here in the module. 

\end{itemize}





















\end {document}